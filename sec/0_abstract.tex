\begin{abstract}
    The global increase in the frequency and intensity of wildfires underscores the need for advancements in fire monitoring techniques. In order to investigate deep learning approaches for detecting and tracking wildfires and the related human health impacts, we present SmokeViz, a large scale machine learning dataset of smoke plumes in satellite imagery. To build the dataset, we refine a set of human-generated annotations created by analysts at the National Oceanic and Atmospheric Administration. Each annotation gives a general temporal and geographical approximation of smoke plumes but at variable and, primarily, low temporal resolution. We present an innovative solution for refining the temporal and spatial resolution in the given analyst annotations by leveraging the semi-supervised method, pseudo-labeling. Unlike typical pseudo-labeling applications that aim to increase the number of labeled samples, the objective is to use pseudo-labels to refine an existing but course-grained set of annotations. We train a deep learning model to generate pseudo-labels that pinpoint the singular, most representative, satellite image to match the smoke annotation within the given temporal range. By identifying the most representative imagery of smoke plumes for a given smoke annotation, the study seeks to create an accurate and relevant machine learning dataset. The resulting SmokeViz dataset is anticipated to be an instrumental tool in developing further machine learning models for studying wildfires and is publically availble at [aws download link].
\end{abstract}
